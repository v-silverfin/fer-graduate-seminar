\documentclass[lmodern, utf8, seminar]{fer}
\usepackage{booktabs}
\usepackage{graphicx}
\graphicspath{ {images/} }

\begin{document}
\nocite{*}



\title{Suparnički generativni modeli za prevođenje slika}

\author{Krešimir Vukić}
\voditelj{prof.~dr.~sc.~Siniša Šegvić}

\maketitle


\tableofcontents



\chapter{Uvod}
Prevođenje slika klasa je problema računalnog vida kojom želimo naučiti funkciju mapiranja između parova slika. Korist te funkcije možemo naći u generiranju novih uzoraka sa željenim zančajkama, identifikaciji postojećih značajki... Jedna od najkorištenijih i najmoćnijih metoda današnjice su suparnički modeli koje ćemo detaljnije razraditi u ovom seminaru.



\chapter{Duboke mreže}
Pristup dobokog učenja posebice je prigodan pri radu sa slikama zahvaljujući njegovoj sposobnosti lokalizacije piksela i slojevitom pristupu pri rastavljanju slike na osnovne građevne jedinice.
\newline
Konvolucija neuronskom mrežom segmentira podatke 



\chapter{Generativni suparnički modeli}
GANovi su podvrsta dubokih mreža čija arhitektura se bazira na dvije podmreže koje se uče kroz međusobno natjecanje. Tako nespecifični model u stanju je kopirati distribuciju širokog spektra podataka: slika, govora, muzike, čak i pisanih zapisa.
\newline
Tip učenja po kojem su dobile naziv je generativne prirode. Problem koji riješavaju je kako dobiti X uz dani Y, tj. pokušavaju naučiti distribuciju pojedinih klasa p(X|Y).
\newline
Pola mreže bavi se stvaranjem novih podataka dok druga polovica pokušava evaluirati koliko su slični stvarnim podatcima.
Cijeli proces usporedljiv je s poslovima detektiav i kriminalca. Jedna strana u bavi se falsifikacijom dokaza, a druga analizom njihove autentičnosti.
\newline
Koraci rada GANa:
 - generator iz nasumičnih brojeva stvara sliku
 - ta slika se dovodi na ulaz diskriminatora uz skup stvarnih slika iz podatkovnog skupa
 - diskriminator određuje postotak vjerojatnosti da je slika prava



\chapter{Uvjetne generativne mreže}
Uvjetne generativne mreže pokazale su se kao generalno rješenje problema prevođenja slike [pix2pix]. Svojstvene su po učenju ne samo funkcije prijenosa između slika već i funkcije gubitka nad kojom se ona trenira.
Dok se u drugim modelima ona ručno zadaje, pristup u kojem se ona uči omogučuje generalizaciju nad setovima problema koji bi u suprotnom zahtijevali vrlo drugačije formulacije gubitka.
\newline
Neke od mogućnosti uvjetnih GAN-ova su sinteza fotorealističnih slika iz skica, generiranje objekata na temelju njihovih obruba te bojanje crno bijelih fotografija.



\chapter{Translacija bez uparenih primjera za učenje pomoću kružnih GAN-ova}




\chapter{Prijenos stila}




\chapter{Zaključak}




\bibliography{literatura}
\bibliographystyle{fer}



\begin{sazetak}
Mnogi problemi u procesiranju slika odnose se na translaciju ulazne slike u željenu izlaznu. Većina današnjih arhitektura bazira se na generativnim suparničkim mrežama, nad kojima su osmišljena mnoga proširenja kako bi se prilagodila specifičnim problemima. Uvjetnim generativnim mrežama nije potrebno definirati funkciju gubitka jer ju mogu same naučiti, kružne generativne mreže sposobne su učenju translacije ne samo ulaznih u izlazne primjere nego i u drugom smjeru, a ograničavanjem mreže na translaciju unutar prostora boja obavljat će prijenos stila bez distorzije sadržaja slike. Ovim radom usporedit ću različite arhitekture, njihove karakteristike i primjene na problemima translacije slike u sliku.

\kljucnerijeci{image-to-image generation, style transfer, GANs, CycleGANs}
\end{sazetak}

\engtitle{Image to image generation using CycleGANs}
\begin{abstract}

\keywords{image-to-image generation, style transfer, GANs, CycleGANs}
\end{abstract}

\end{document}