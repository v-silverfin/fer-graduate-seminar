\documentclass[lmodern, utf8, seminar]{fer}
\usepackage{booktabs}
\usepackage{graphicx}
\graphicspath{ {images/} }

\begin{document}
\nocite{*}



\title{Suparnički generativni modeli za prevođenje slika}

\author{Krešimir Vukić}
\voditelj{prof.~dr.~sc.~Siniša Šegvić}

\maketitle


\tableofcontents



\chapter{Poglavlja seminara}
TODO
\chapter{Uvod}
TODO
\chapter{Duboke mreže}
TODO
\chapter{Generativni suparnički modeli}
TODO
\chapter{Uvjetne generativne mreže}
TODO 
\chapter{Translacija bez uparenih primjera za učenje pomoću kružnih GAN-ova}
TODO
\chapter{Prijenos stila}
TODO



\chapter{Zaključak}
TODO



\bibliography{literatura}
\bibliographystyle{fer}



\begin{sazetak}
Mnogi problemi u procesiranju slika odnose se na translaciju ulazne slike u željenu izlaznu. Većina današnjih arhitektura bazira se na generativnim suparničkim mrežama, nad kojima su osmišljena mnoga proširenja kako bi se prilagodila specifičnim problemima. Uvjetnim generativnim mrežama nije potrebno definirati funkciju gubitka jer ju mogu same naučiti, kružne generativne mreže sposobne su učenju translacije ne samo ulaznih u izlazne primjere nego i u drugom smjeru, a ograničavanjem mreže na translaciju unutar prostora boja obavljat će prijenos stila bez distorzije sadržaja slike. Ovim radom usporedit ću različite arhitekture, njihove karakteristike i primjene na problemima translacije slike u sliku.

\kljucnerijeci{image-to-image generation, style transfer, GANs, CycleGANs}
\end{sazetak}

\engtitle{Image to image generation using CycleGANs}
\begin{abstract}

\keywords{image-to-image generation, style transfer, GANs, CycleGANs}
\end{abstract}

\end{document}